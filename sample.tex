%!TEX TS-program = lualatex
\documentclass[aspectratio=169]{beamer}

% Minimalistické nastavení tématu
\usepackage{tikz}
\usetheme{focusdeck}
\usepackage{ragged2e}
\usepackage{fontspec}
\setmainfont{TeX Gyre Heros}
\setsansfont{TeX Gyre Heros}
% \usepackage[sfdefault]{tgheros}



% Titulní snímek
\title{Název Prezentace}
\author{Autor}
\institute{Instituce}
\date{\today}

% Obsah prezentace
\begin{document}

% Titulní strana
\begin{frame}
    \titlepage
\end{frame}

% Osnova
\begin{frame}{Obsah}
    \tableofcontents
\end{frame}

% Sekce a snímky
\section{Úvod}
\begin{frame}{Úvod}
    Toto je úvodní snímek.
    \begin{block}{Hlavní myšlenka}
        Tady je blok textu, který prezentuje hlavní myšlenku.
    \end{block}
    \begin{itemize}
        \item První bod
        \item Druhý bod
    \end{itemize}
\end{frame}

\section{Hlavní část}
\begin{frame}{Hlavní část}
    Tento snímek obsahuje hlavní část prezentace.
    \begin{alertblock}{Důležité upozornění}
        Text důležitého upozornění nebo poznámky.
    \end{alertblock}
    \begin{itemize}
        \item První bod
        \item Druhý bod
    \end{itemize}
\end{frame}

\begin{frame}[fragile]{Další slide}
 Tady  je nějakej text
\end{frame}


\section{Závěr}
\begin{frame}{Závěr}
    Shrnutí a závěr prezentace.
    \begin{exampleblock}{Příklad}
        Zde je blok s příkladem nebo dodatečným obsahem.
    \end{exampleblock}
\end{frame}

% Konec prezentace
\begin{frame}
    \centering
    \Huge Děkuji za pozornost!
\end{frame}


\end{document}
